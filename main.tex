\documentclass{article}
\usepackage{amsmath, amssymb, amsthm}
\usepackage[utf8]{inputenc}
\usepackage[spanish]{babel}

\title{Teorema de Necesidad y Suficiencia para la Integrabilidad de Riemann}
\author{}
\date{}

\begin{document}

\maketitle

\section*{Introducción}
En el análisis matemático, el concepto de integrabilidad de Riemann es fundamental para definir la integral de una función en un intervalo cerrado $[a, b]$. Una función $f: [a, b] \to \mathbb{R}$ se dice Riemann integrable si existe un número $I$ tal que, para todo $\epsilon > 0$, existe una partición $P$ de $[a, b]$ para la cual las sumas superiores e inferiores de $f$ sobre $P$ están suficientemente cerca de $I$. Este teorema establece condiciones necesarias y suficientes para que una función sea Riemann integrable, y se puede expresar de tres maneras equivalentes:

\begin{enumerate}
    \item \textbf{Condición de Riemann}: Para todo $\epsilon > 0$, existe una partición $P$ de $[a, b]$ tal que la diferencia entre la suma superior $U(P, f)$ y la suma inferior $L(P, f)$ es menor que $\epsilon$.
    
    \item \textbf{Condición de Darboux}: El límite de las sumas superiores $U(P, f)$ y el límite de las sumas inferiores $L(P, f)$ coinciden cuando el tamaño de la partición tiende a cero.
    
    \item \textbf{Condición de Cauchy}: Para todo $\epsilon > 0$, existe una partición $P$ de $[a, b]$ tal que, para cualquier refinamiento $P'$ de $P$, la diferencia entre las sumas superior e inferior sobre $P'$ es menor que $\epsilon$.
\end{enumerate}

En este texto, demostraremos rigurosamente que estas tres condiciones son equivalentes, es decir, que cada una implica a las otras dos. Esto se hará demostrando las siguientes implicaciones:

\begin{enumerate}
    \item La condición de Riemann implica la condición de Darboux.
    \item La condición de Darboux implica la condición de Cauchy.
    \item La condición de Cauchy implica la condición de Riemann.
\end{enumerate}

Con estas tres implicaciones, habremos establecido la equivalencia completa entre las tres condiciones.

\section*{Demostración de las Implicaciones}

\subsection*{1. La condición de Riemann implica la condición de Darboux}

\textbf{Hipótesis}: Supongamos que se cumple la condición de Riemann. Es decir, para todo $\epsilon > 0$, existe una partición $P$ de $[a, b]$ tal que:
\[
U(P, f) - L(P, f) < \epsilon.
\]

\textbf{Tesis}: Demostraremos que se cumple la condición de Darboux, es decir:
\[
\lim_{||P|| \to 0} U(P, f) = \lim_{||P|| \to 0} L(P, f).
\]

\textbf{Demostración}:

Dado $\epsilon > 0$, por la condición de Riemann, existe una partición $P$ tal que:
\[
U(P, f) - L(P, f) < \epsilon.
\]

Como $U(P, f)$ es una cota superior para las sumas superiores y $L(P, f)$ es una cota inferior para las sumas inferiores, se tiene:
\[
L(P, f) \leq \int_a^b f(x) \, dx \leq U(P, f).
\]

Por lo tanto, $U(P, f) - L(P, f) < \epsilon$ implica que:
\[
|U(P, f) - \int_a^b f(x) \, dx| < \epsilon \quad \text{y} \quad |L(P, f) - \int_a^b f(x) \, dx| < \epsilon.
\]

Esto muestra que $U(P, f)$ y $L(P, f)$ convergen al mismo valor (la integral de $f$) cuando $||P|| \to 0$.

Por lo tanto, se cumple la condición de Darboux.

\subsection*{2. La condición de Darboux implica la condición de Cauchy}

\textbf{Hipótesis}: Supongamos que se cumple la condición de Darboux. Es decir:
\[
\lim_{||P|| \to 0} U(P, f) = \lim_{||P|| \to 0} L(P, f).
\]

\textbf{Tesis}: Demostraremos que se cumple la condición de Cauchy, es decir, para todo $\epsilon > 0$, existe una partición $P$ de $[a, b]$ tal que, para cualquier refinamiento $P'$ de $P$, se cumple:
\[
U(P', f) - L(P', f) < \epsilon.
\]

\textbf{Demostración}:

Dado $\epsilon > 0$, por la condición de Darboux, existe $\delta > 0$ tal que, para cualquier partición $P$ con $||P|| < \delta$, se cumple:
\[
|U(P, f) - L(P, f)| < \epsilon.
\]

Sea $P$ una partición con $||P|| < \delta$.

Para cualquier refinamiento $P'$ de $P$, se tiene $||P'|| \leq ||P|| < \delta$.

Por lo tanto, $U(P', f) - L(P', f) < \epsilon$.

Esto muestra que se cumple la condición de Cauchy.

\subsection*{3. La condición de Cauchy implica la condición de Riemann}

\textbf{Hipótesis}: Supongamos que se cumple la condición de Cauchy. Es decir, para todo $\epsilon > 0$, existe una partición $P$ de $[a, b]$ tal que, para cualquier refinamiento $P'$ de $P$, se cumple:
\[
U(P', f) - L(P', f) < \epsilon.
\]

\textbf{Tesis}: Demostraremos que se cumple la condición de Riemann, es decir, para todo $\epsilon > 0$, existe una partición $P$ de $[a, b]$ tal que:
\[
U(P, f) - L(P, f) < \epsilon.
\]

\textbf{Demostración}:

Dado $\epsilon > 0$, por la condición de Cauchy, existe una partición $P$ tal que, para todo refinamiento $P'$ de $P$, se cumple:
\[
U(P', f) - L(P', f) < \epsilon.
\]

En particular, $P$ es un refinamiento de sí misma, por lo que:
\[
U(P, f) - L(P, f) < \epsilon.
\]

Esto muestra que se cumple la condición de Riemann.

\section*{Conclusión}
Hemos demostrado rigurosamente las tres implicaciones:

\begin{enumerate}
    \item La condición de Riemann implica la condición de Darboux.
    \item La condición de Darboux implica la condición de Cauchy.
    \item La condición de Cauchy implica la condición de Riemann.
\end{enumerate}

Por lo tanto, las tres condiciones son equivalentes:
\[
\text{Condición de Riemann} \iff \text{Condición de Darboux} \iff \text{Condición de Cauchy}.
\]

Esto establece que una función $f$ es Riemann integrable en $[a, b]$ si y solo si se cumple cualquiera de estas tres condiciones. Esta equivalencia es fundamental en el análisis matemático y proporciona una comprensión profunda de la integrabilidad de Riemann.

\end{document}